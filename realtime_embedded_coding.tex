\documentclass[12pt]{report}
\usepackage{caption}
\usepackage{graphicx}
\usepackage{hyperref}
\hypersetup{%
    pdfborder = {0 0 0}
}
\hypersetup{
    colorlinks,
    citecolor=blue,
    filecolor=blue,
    linkcolor=blue,
    urlcolor=blue
}
\renewcommand{\familydefault}{\sfdefault}
\renewcommand{\captionfont}{\small}

\author{Bernd Porr \& Nick Bailey}
\title{Realtime embedded coding in C++ under Linux}

\begin{document}

\maketitle

\tableofcontents

\chapter{Introduction}



\begin{figure}[!hbt]
\begin{center}
\mbox{\includegraphics[width=\textwidth]{signals-timings}}
\end{center}
\caption{Dataflow and timing in low level realtime coding
\label{timing}}
\end{figure}

Realtime embedded coding is all about \textsl{events}.
These can be a binary signal such as somebody opening a door or
an ADC signalling that a sample is ready to be picked up.
Fig.~\ref{timing} shows the basic dataflow and how event timing is
established: devices by themselves have event signals such as ``data
ready'' or ``crash sensor has been triggered''. The Linux kernel receives
these as interrupt callbacks. However, userspace has no direct interrupt
mechanism; instead it has blocking I/O where a read or write operation blocks
until a kernel-side interrupt has happened. A blocking I/O call returning
may then be translated callbacks between classes by waking up threads.
Data is transmitted back to the hardware via methods called ``setters'',
which change an object's attributes and potentially do other processing.

\begin{figure}[!hbt]
\begin{center}
\mbox{\includegraphics[width=\textwidth]{gettersetters}}
\end{center}
\caption{A realtime system with two C++ classes. Communication
  between classes is achieved with callbacks (not getters) for incoming events
  and setters to send out control events. The control output itself
  receives its timing from the events so that the loop is traversed
  as quickly as possible.
\label{gettersetters}}
\end{figure}
Fig.~\ref{gettersetters} shows the overall communication between C++
classes in a realtime system. This communication is done via callbacks
(\textsl{not} getters) and setters, where an event from a sensor
traverses according to its realtime requirements through the C++ classes via
callbacks and then back to the control output via setters. For example,
a collision sensor at a robot triggers a GPIO pin, which then triggers a
callback to issue an avoidance action which in turn then sets the
motors in reverse.

When developing the C++ classes keep 
\href{https://www.digitalocean.com/community/conceptual_articles/s-o-l-i-d-the-first-five-principles-of-object-oriented-design}{S.O.L.I.D.}
in mind:
\begin{description}
\item[Single responsibility:] If you have a temperature
sensor and an accelerometer then write two classes: one for the
temperature sensor and one of the accelerometer. Don't write one
class \texttt{Hardware} which takes an argument \texttt{Hardware::temp}
or \texttt{Hardware::accel} and then does one of two completely different
things. It's a debugging nightmare and wasteful if you want to reuse
your class in hardware with \emph{only} a temperature sensor \emph{or}
an accelerometer.
\item[Open-Closed principle:] Your class is open to extension but
  closed to modification. For example an ADC class
  has a callback which returns voltage to the client. However,
  you'll be connecting, for example, a temperature sensor to
  it, so you'd like to be able to extend the class
  overloading the callback methods so that you add the conversion
  from volt to degrees. This is a Good Thing\texttrademark{} so long as
  you create a derived class. It's a bad idea to hack the existing
  ADC class adding a \texttt{to\_kelvin} method: why would somebody
  using the ADC to read a value from an accelerometer need that?
\item[Liskov substitution principle:] Strictly, substituting
  a derived class for its base class does not result in the program
  becoming incorrect. That is to say, any derived class from
  your device driver class can be used in place of the base class if
  the base class is all that's required, because the extra
  functionality in the derived class shouldn't break the basic
  required functionality of the base class. For example, if you have a
  super-duper DAC with lots of extra features, it shouldn't stop you
  using it when you only need a very simple one. This also means
  that sensible default values should be set so that the client
  won't need to understand the nerdy features of that super-duper DAC.
\item[Interface Segregation principle:]
  Keep functionality separate and aim to divide it up in different
  classes. Imagine you have a universal IO class with SPI and I2C
  but your client really just needs SPI. Then the client is forced
  to deactivate I2C or in the worst case the class causes collateral
  damage without the client knowing why.
\item[Dependency inversion:] That is about abstracting the
  essential features of a class of interfaces. For example, ideally
  you want a base class covering a range of similar
  ADC converters from the same manufacturer and not a base class being
  a driver for a particular chip. Individual ADC chip driver
  classes then inherit from the abstract ADC driver. The authors of
  the ADC's driver will no doubt consider their chip's
  particular capabilities to be the core ideas,
  but this needs inverting. Why would one ADC driver need
  to provide all the necessaray code for a different one of the
  same family? Instead, the driver is dependent on the abstract
  idea of an ADC, not the other way around. The concrete depends
  on the abstract.
\end{description}


Besides S.O.L.I.D it's also essential that:
\begin{enumerate}
\item the project has a \textbf{build system} such as cmake. It's
  strongly recommended to use \textbf{cmake} (autoconf only for older existing
  projects). This is especially true for Qt projects, where cmake support
  is a stated priority of the Qt developers.
\item classes (in particular driver classes) are \textbf{re-usable}
  outwith of the specific project and have
  their own cmake-projects in their subdirectories. Except for the main
  cmake project all sub-projects are libraries. This aids testing and
  reuse.
\item all public interfaces have \textbf{doc-strings} for all public
  methods/constants and an automatically generated reference, for
  example with the help of \textbf{doxygen}.
\item classes which perform internal processing such as filters,
  databases, detectors, \ldots, have \textbf{unit tests} and are run via
  the cmake testing framework.
\item the documentation provides comprehensive information about the project itself,
  how to install and run the project.
\end{enumerate}


\chapter{Writing C++ device driver classes}
This chapter focuses on writing your own C++ device driver class
hiding away the complexity (and messy) low level C APIs and/or raw
device access. How are events translated into I/O operations? On the
hardware-side we have event signals such as data-ready signals or by
the timing of a serial or audio interface. The Linux kernel translates
this timing info into blocking I/O on pseudo filesystems such as /dev
or /sys which means that a read operation blocks until data has arrived
or an event has happened. Some low level libraries such as \textbf{pigpio}
translate them back into C callbacks. The task of a C++ programmer is
to hide this complexity and these quite different approaches in C++
classes which communicate via callbacks and setters with the client
classes. The rest of the program will then appear simple and be
easy to maintain.

\section{General recommendations on how to write your C++ classes for devices}
As said above the main purpose of object oriented coding here is to
hide away the complexity of low level driver access and offer the
client a simple and safe way of connecting to the sensor. In
particular:
\begin{enumerate}
\item Setters and callbacks hand over \textsl{physical units}
  (temperature, acceleration, \ldots) or relative units but not raw
  integer values with no meaning.
\item The sensor is configured by specifying physical units (time,
  voltage, temperature) and not sensor registers. Default config parameters
  should be specified that the class can be used straight away with
  default parameters.
\item The class comes with simple demo programs demonstrating how
  a client program might use it.
\end{enumerate}

\section{Low level userspace device access}
The following sections provide pointers of how to write
the C++ driver classes for different hardware protocols.

\subsection{SPI}
\begin{table}[!ht]
  \begin{center}
  \caption{SPI modes\label{spimodes}}
  \begin{tabular}{l|l|l|l}
    SPI Mode & 	CPOL & 	CPHA & Idle state \\
    \hline
    0& 	0&	0& 	L \\
    1& 	0&	1& 	L \\
    2& 	1&	1& 	H \\
    3& 	1&	0& 	H \\
  \end{tabular}
  \end{center}
\end{table}
SPI is a protocol which usually transmits and receives at the same
time. Even though data might not be used it needs to be matched up,
because the same clock is used to send and collect the data signal
(it is \emph{isochronous}).
So if you send 8 bytes the hardware receives 8 bytes at the same time.

Transfer to/from SPI is best managed by the low level access to \texttt{/dev}.
Open the SPI device with the standard \texttt{open()} function:
\begin{verbatim}
int fd = open( "/dev/spidev0.0", O_RDWR);
\end{verbatim}

Then set the SPI mode (see table~\ref{spimodes}):
\begin{verbatim}
int ret = ioctl(fd, SPI_IOC_WR_MODE, &mode);
\end{verbatim}
as explained, for example, here:
\url{https://www.analog.com/en/analog-dialogue/articles/introduction-to-spi-interface.html}.

Because SPI is isochronous, \texttt{read()} and \texttt{write()}
can't be used to transmit and receive data. Instead, the simultaneous
read and write is performed using an \texttt{ioctl()} to do the communication.
Populate this struct:
\begin{verbatim}
struct spi_ioc_transfer tr { % leave out the = ? It's 2022...
  .tx_buf = (unsigned long)tx1,
  .rx_buf = (unsigned long)rx1,
  .len = ARRAY_SIZE(tx1),
  .delay_usecs = delay,
  .speed_hz = speed,
  .bits_per_word = 8,
};
\end{verbatim}
which points to two character buffers ``tx'' and ``rx'' with the
same length. (Note: this code fragment's use of designated initialisers
officially requires C++2a, although most C++ compilers support it
when compiling with older standards. In C it's been fine for a while!)

Reading and simultaneous writing then happens via this \texttt{ioctrl()}
function call:
\begin{verbatim}
int ret = ioctl(fd, SPI_IOC_MESSAGE(1), &tr);
\end{verbatim}

Sometimes the SPI protocol of a chip is so odd that even the raw
I/O via \texttt{/dev} won't work and you need to write your own bit banging
interface, for example done here for the ADC on the alphabot:
\url{https://github.com/berndporr/alphabot/blob/main/alphabot.cpp#L58}.
This is obviously far from ideal as it might require \texttt{usleep()} commands
so that acquisition needs to be run in a separate thread (the alphabot indeed
uses a timer callback in a separate thread).

Overall the SPI protocol is often device dependent and calls
for experimentation to get it to work. On some ADCs the SPI clock is also
the conversion clock and a longer lasting clock signal sequence is required,
making it necessary to transmit dummy bytes in addition to the payload.

As a general recommendation do not use SAR converters which use the
SPI data clock also as acquisition clock as they are often not compatible
with the standard SPI transfers via \texttt{/dev}. Use sensors or ADCs which
have their own clock signal.


\subsection{I2C}
The I2C bus has two signal lines (SDA \& SDL) which must be pulled up
by resistors. Every I2C device has an address on the bus. You can scan
a bus with \textbf{i2cdetect} (part of the i2c-tools package):
\begin{verbatim}
root@raspberrypi:/home/pi# i2cdetect -y 1
     0  1  2  3  4  5  6  7  8  9  a  b  c  d  e  f
00:                         -- -- -- -- -- -- -- -- 
10: -- -- -- -- -- -- -- -- -- -- -- -- -- -- 1e -- 
20: -- -- -- -- -- -- -- -- -- -- -- -- -- -- -- -- 
30: -- -- -- -- -- -- -- -- -- -- -- -- -- -- -- -- 
40: -- -- -- -- -- -- -- -- -- -- -- -- -- -- -- -- 
50: -- -- -- -- -- -- -- -- 58 -- -- -- -- -- -- -- 
60: -- -- -- -- -- -- -- -- -- -- -- 6b -- -- -- -- 
70: -- -- -- -- -- -- -- --                         
root@raspberrypi:/home/pi# 
\end{verbatim}
In this case there are 3 I2C devices on the I2C bus at addresses
1E, 58 and 6B and need to be specified when
accessing the I2C device.

\subsubsection{Raw \texttt{/dev/i2c} access}
I2C either transmits or receives but never at the same time so here we
can use the standard C read/write commands. However, we need to use ioctrl to tell
the kernel the I2C address:
\begin{verbatim}
char buf[2];
int file = open("/dev/i2c-2",O_RDWR);
int addr = 0x58;
ioctl(file, I2C_SLAVE, addr);
write(file, buf, 1)
read(file, buf, 2)
\end{verbatim}
where \texttt{addr} is the I2C address. Then use standard \texttt{read()}
or \texttt{write()} commands. Usually the 1st write operation tells the chip
which register to read or write to. Subsequent operations read
or write that register.

\subsubsection{I2C access via pigpio}
Access via pigpio (\url{http://abyz.me.uk/rpi/pigpio/cif.html})
is preferred in contrast to direct
access of the raw /dev/i2c because many different devices
can be connected to the I2C bus and pigpio manages this.
Simply install the development package:
\begin{verbatim}
sudo apt-get install libpigpio-dev
\end{verbatim}
which triggers then the install of the other relevant packages.
For example writing a byte to a register in an I2C sensor can be done with a
few commands:
\begin{verbatim}
int fd = i2cOpen(i2c_bus, address, 0);
i2cWriteByteData(fd, subAddress, data);
i2cClose(fd);
\end{verbatim}
where \texttt{i2c\_bus} is the I2C bus number (usually 1 on the RPI)
and the \texttt{address} is the I2C address of the device on that bus.
The \texttt{subAddress} here is the register address in the device.

\subsection{Access GPIO pins}
\subsubsection{\texttt{/sys} filesystem}
The GPIO of the raspberry PI can easily be controlled via
the \texttt{/sys} filesystem. This is slow but good for
debugging as you can directly write a
``0'' or ``1'' string to it and print the result. The
pseudo files are here:
\begin{verbatim}
/sys/class/gpio
\end{verbatim}
which contains files which directly relate to individual pins.
To be able to access a pin we need to tell Linux to make
it visible:
\begin{verbatim}
/sys/class/gpio/export
\end{verbatim}
For example, writing a 5 (in text form) to this file would
create the subdirectory \texttt{/sys/class/gpio/gpio5} for GPIO pin 5.
Then reading from
\begin{verbatim}
/sys/class/gpio/gpio5/value
\end{verbatim}
would give you the status of GPIO pin 5 and writing
to it would change it.
A thin wrapper around the GPIO sys filesystem is here: \url{https://github.com/berndporr/gpio-sysfs}.

\paragraph{GPIO interrupt handling via \texttt{/sys}\label{gpioIRQ}.}
The most important application for the /sys filesystem is to
do interrupt processing in userspace.
A thread can be put to sleep until an interrupt has happened on one of
the GPIO pins. This is done by monitoring the ``value''
of a GPIO pin in the \texttt{/sys} filesystem with the ``poll'' command:
\begin{verbatim}
struct pollfd fdset[1];
int nfds = 1;
int gpio_fd = open("/sys/class/gpio/gpio5/value", O_RDONLY | O_NONBLOCK );
memset((void*)fdset, 0, sizeof(fdset));
fdset[0].fd = gpio_fd;
fdset[0].events = POLLPRI;
int rc = poll(fdset, nfds, timeout);
if (fdset[0].revents & POLLPRI) {
   // dummy
   read(fdset[0].fd, buf, MAX_BUF);
}
\end{verbatim}
This code makes the thread go to sleep until an interrupt has occurred on
GPIO pin~5. Then the thread wakes up and execution continues.

\subsubsection{pigpio}
The above section has given you a deep understanding what's happening
under the hood on the sysfs-level but it's highly recommended to
use the pigpio library (\url{http://abyz.me.uk/rpi/pigpio/cif.html})
to read/write to GPIO pins or do interrupt programming.

For example to set GPIO pin 24 as an input just call:
\begin{verbatim}
gpioSetMode(24,PI_INPUT);
\end{verbatim}

To read from GPIO pin 24 just call:
\begin{verbatim}
int a = gpioRead(24)
\end{verbatim}

\paragraph{interrupt handling via pigpio.}
pigpio manages GPIO interrupt handling by wrapping all the above
functionality into a single command where the client registers a
callback function. The callback occurs whenever a GPIO pin changes.
Specifically a method of the form:
\begin{verbatim}
class mySensorClass {
  ...
  static void gpioISR(int gpio, int level, uint32\_t tick, void* userdata)
  ...
}
\end{verbatim}
is registered with pigpio:
\begin{verbatim}
gpioSetISRFuncEx(24,RISING_EDGE,ISR_TIMEOUT,gpioISR,(void*)this);
\end{verbatim}
where ``this'' is the pointer to your class instance.
The callback registered will then be \texttt{this->dataReady()}.
\begin{verbatim}
class LSM9DS1 {
  void dataReady();
  static void gpioISR(int gpio, int level, uint32_t tick, void* userdata)
    {
        ((LSM9DS1*)userdata)->dataReady();
    }
};
\end{verbatim}
where here within the static function the void pointer is cast back into the instance pointer.
See \url{https://github.com/berndporr/LSM9DS1_RaspberryPi_CPP_Library} for the complete code.



\subsection{Access to hardware via special devices in \texttt{/sys}}
Some sensors are directly available via the sys filesystem in human readable format.
For example
\begin{verbatim}
cat /sys/class/thermal/thermal_zone0/temp
\end{verbatim}
gives you the temperature of the CPU.




\subsection{I2S: Audio}
The standard framework for audio is alsa: \url{https://github.com/alsa-project}.

ALSA is packet based with a read command
returning a chunk ("buffer") of audio and write emitting one.
There are calls to set the sample format, sample rate, buffer size
and so forth.

First, the parameters are requested and the driver can modify or
reject them:
\begin{verbatim}
/* Signed 16-bit little-endian format */
  snd_pcm_hw_params_set_format(handle, params,
                               SND_PCM_FORMAT_S16_LE);

  /* One channel (mono) */
  snd_pcm_hw_params_set_channels(handle, params, 1);

  /* 44100 bits/second sampling rate (CD quality) */
  val = 44100;
  snd_pcm_hw_params_set_rate_near(handle, params,
                                  &val, &dir);
\end{verbatim}

Then playing sound is done in an endless loop were a read()
or write() command is issued. Both are blocking so that
it needs to run in a thread:

\begin{verbatim}
 while (running) {
   if ((err = snd_pcm_readi (handle, buffer, buffer_frames)) != buffer_frames) {
     if (errCallback) errCallback->hasError();
   }
   if (sampleCallback) sampleCallback->hasData(buffer);
 }
\end{verbatim}

For a full coding example ``aplay'' and ``arecord'' are a good start.
Both can be found here:
\url{https://github.com/alsa-project}.




\subsection{Accessing physical memory locations (danger!)}

Don't.
In case you really need to access registers you can
also access memory directly. This should only be used as a last resort.
For example, setting the clock for the AD converter requires
turning a GPIO pin into a clock output pin. This is not yet
supported by the drivers so we need to program registers
on the RPI.
\begin{itemize}
\item Linux uses virtual addresses so that a pointer won't
point to a physical memory location. It points to three page
tables with an offset.
\item Special device \texttt{/dev/mem} allows access to physical
memory.
\item The command \textbf{mmap} provides a pointer to a physical
address by opening \texttt{/dev/mem}.
\item Example:
\begin{verbatim}
int *addr;
if ((fd = open("/dev/mem", O_RDWR|O_SYNC)) < 0 ) {
    printf("Error opening file. \n");
    close(fd);
    return (-1);
}
addr = (int *)mmap(0, num*STRUCT_PAGE_SIZE, PROT_READ, MAP_PRIVATE,
            fd, 0x0000620000000000);
printf("addr: %p \n",addr);
printf("addr: %d \n",*addr);
\end{verbatim}
\item Use this with care! It's dangerous if not used properly.
\end{itemize}


\section{Kernel driver programming}
You can also create your own \texttt{/dev/mydevice} in the \texttt{/dev} filesystem
by writing a kernel driver and a matching userspace library. For
example the USB mouse has a driver in kernel space and translates
the raw data from the mouse into coordinates. However,
this is beyond the scope of this handout. If you want to embark
on this adventure then the best approach is to
find a kernel driver which does approximately what you want and
modify it for your purposes.


\section{Callbacks in C++ device classes (getting data)}
As said in the introduction your hardware device class has callback interfaces
to hand back the data to the client.

There are different ways of tackling the issue of callbacks but the
simplest one is defining a method as \textsl{abstract} and asking the
client to implement it in a derived class. That abstract function can
either be in a separate interface class or part of the device class
itself. So, we have two options:
\begin{enumerate}
\item The callback is part of the device driver class:
\begin{verbatim}
class MyDriver {
          void start(DevSettings settings = DevSettings() );
          void stop();
          virtual void callback(float sample) = 0;
};
\end{verbatim}
\item The callback is part of an interface class:
\begin{verbatim}
class CallbackInterface {
          virtual void callback(float sample) = 0;
};
\end{verbatim}
and then registering it in the main device driver class:
\begin{verbatim}
class MyDriver {
          void registerCallback(CallbackInterface* cb);
};
\end{verbatim}
\end{enumerate}
These two options are now explained in greater detail.


\subsection{Creating a callback interface}
Here, we create a separate interface class containing a callback
as an abstract method:
\begin{verbatim}
class LSM9DS1callback {
public:
        virtual void hasSample(LSM9DS1Sample sample) = 0;
};
\end{verbatim}

The client then implements the abstract method \texttt{hasSample()}, instantiates
the interface class and then saves its pointer in the device class, here called \texttt{lsm9ds1Callback}.
\begin{verbatim}
void LSM9DS1::dataReady() {
        LSM9DS1Sample sample;
        // fills the sample struct with data
        // ...
        lsm9ds1Callback->hasSample(sample);
}
\end{verbatim}
The pointer to the interface instance is transmitted via a setter which
receives the pointer of the interface as an argument, for example:
\begin{verbatim}
        void registerCallback(LSM9DS1callback* cb);
\end{verbatim}
This allows to register a callback optionally. The client may or may not need
one.
See
\url{https://github.com/berndporr/rpi_AD7705_daq}
for a complete example.

\subsection{Adding directly an abstract method to the device driver class}
Instead of creating a separate class containing the callback you
can also add the callback straight to the device driver class.
\begin{verbatim}
class ADS1115rpi {
        ...
        virtual void hasSample(float sample) = 0;
        ...
};
\end{verbatim}
This forces the client to implement the callback to be able to use
the class. This creates a very safe environment as all dependencies
are set at compile time and the abstract nature of the base class
makes clear what needs to be implemented.
See
\url{https://github.com/berndporr/rpi_ads1115} for a complete example.

\subsection{Callback arguments}
Above the callbacks just delivered one floating point value. However,
often more than one sample or more complex data are transmitted:
\begin{itemize}
\item Complex data: do not put loads of arguments into the
  callback but define a \textsl{struct}. For example an ADC might
  deliver all 4 channels at once:
\begin{verbatim}
class ADmulti {

        struct ADCSample {
            float ch1;
            float ch2;
            float ch3;
            float ch4;
        };

        ...
        virtual void hasSample(ADCSample sample) = 0;
        ...
};
\end{verbatim}
Depending on your application,
you might consider the values not useful individually and therefore prefer
a \texttt{std::tuple}.
\item Arrays: Use arrays which contain the length of the arrays:
  either std::array, std::vector, etc or const arrays and then
  references to these so that the callback knows the length.
  For example the LIDAR callback uses a reference to a const length
  array:
\begin{verbatim}
/**
 * Callback interface which needs to be implemented by the user.
 **/
struct DataInterface {
        virtual void newScanAvail(
                float rpm, 
                A1LidarData (&)[A1Lidar::nDistance]) = 0;
};
\end{verbatim}
Here \texttt{A1Lidar::nDistance} is a constant-expression giving the fixed array length,
and \texttt{(\&)[A1Lidar::nDistance]} a reference to a constant-length array which
containing \texttt{A1LidarData} stucts. If you're going to use types that hard to key-in
often, \texttt{typedef} them.
\end{itemize}
In terms of \textsl{memory management}:
\begin{enumerate}
\item Low sampling rate complex data structures: allocate as a local variable. It can be a simple type
  or a struct. See \texttt{dataReady()} in: \url{https://github.com/berndporr/LSM9DS1_RaspberryPi_CPP_Library/blob/master/LSM9DS1.cpp}.
\item High sampling rate buffers: allocate memory on the heap in the
  constructor or in the private section of the class as a const length
  array and pass on a \textsl{reference}. See \texttt{getData()} in
  \url{https://github.com/berndporr/rplidar_rpi}.
\end{enumerate}


\section{Conclusion}
The communication between C++ classes is achieved via
callbacks and setters. The event from the sensor traverses
the C++ classes via callbacks and then back to the control
output via setters.

From the sections above it's clear that Linux user-space low level
device access is complex, even without taking into account the
complexity of contemporary chips which have often a multitude of
registers and pages of documentation. Your task is to hide away
all this (scary) complexity in a C++ class and offer the client
an easy-to-understand interface.





\chapter{Threads}

\section{Introduction}
In realtime systems threads have two distinct functions:
\begin{enumerate}
\item Endless loops with blocking I/O or GPIO wakeups
  to establish \textbf{precise timing} for callbacks.
\item \textbf{Asynchronous execution} of time-consuming tasks
  with a callback after the task has completed.
\end{enumerate}


\section{Processes and Threads}
Processes are different programs which seem to be running at the same
time. A small embedded system may only have a single CPU core,
so this this is achieved by the operating system switching
approximately every 10ms from one process to the next so it feels as
if they are running concurrently. A thread is a lightweight
process. A process may have  multiple threads which share the same
address-space and are all started from
within the parent process. As with processes, the threads seem to be
running at the same time. When a thread is started it runs
simultaneously with the main process which created it.

\section{Thread and worker}
A thread is just a \textsl{wrapper} for the actual method
which is running independently. The method being run in the thread
is often called a \textsl{worker}.

\subsection{Creating threads}
In C++ a worker is a method within
a class and needs to be \textsl{static} which means it won't be
able to access the instance variables of a class. The trick
is to pass a pointer to the instance of the class (\texttt{this}) as the argument of
the worker, in the following example called \texttt{exec}:
\begin{verbatim}
uthread = new std::thread(MyClassWithAThread::exec, this);
\end{verbatim}
where \texttt{MyClassWithAThread} is a class containing the static function ``exec'':
\begin{verbatim}
class MyClassWithAThread {
        void run() {
                // ... hard work is done here
                doCallback(result); // hand the result over
        }
        static void exec(MyClassWithAThread* cppThread) {
                cppThread->run();
        }
}
\end{verbatim}
which in turn then calls a non-static class method \texttt{run()}.
\texttt{run} will then have access to all of the instance's attributes
and methods.

\subsection{Lifetime of a thread}
Threads terminate simply once the static worker has finished its job.
To tell the client that a thread has finished you can use a
\textsl{callback} to trigger an event.

Sometimes it's important to wait for the termination of the thread,
for example when your whole program is terminating or when
you stop an endless loop in a thread. To wait for the termination
of the thread use the ``join()'' method:
\begin{verbatim}
        void stop() {
                uthread->join();
                delete uthread;
        }
\end{verbatim}
It's also important is also to release the memory of a thread after it has finished
to avoid memory leaks, hence the \texttt{delete} command.


\subsection{Running/stopping workers with endless loops}
Threads with endless loops are often used in conjunction with blocking
I/O which provide the timing:
\begin{verbatim}
void run() {
       running = true;
       while (running) {
              read(buffer); // blocking
              doCallback(buffer); // hand data to client
       }
}
\end{verbatim}
Note the flag \texttt{running} which is controlled by the main program and is set to zero to terminate
the thread:
\begin{verbatim}
        void stop() {
                running = false; // <----- HERE!!
                uthread->join();
                delete uthread;
        }
\end{verbatim}
Note that \texttt{join()} is a blocking operation and needs to be used with care not to
lock up the main program. You probably only need it when your program is terminating.
See \url{https://github.com/berndporr/rpi_AD7705_daq} for an example.

If your program creates and joins several threads while executing,
with care you might be able to design your program to allow an
end-of-life thread to carry on
tidying up in the background after you stop it by setting \texttt{running = false;}.
In this case only need execute \texttt{join()} to be sure that
the thread has finished. Joining a terminated thread is OK
(the call just returns immediately)
but you absolutely  must not join a thread more than once,
nor delete a thread until you've joined it.


\subsection{Timing within threads}
Threads are perfect to create timing without using sleep commands
with the help of \textsl{blocking I/O}.

\subsubsection{select/poll commands waiting for GPIO interrupts}
In section~\ref{gpioIRQ} we introduced the so called ``poll'' command
which is not polling an IRQ pin but \textsl{putting a thread to sleep} till an
external event has happened. Then of course a callback function should
be called reacting to the external event. This is the preferred method
for low latency responses.

As said previously, use \textbf{pigpio} on the Raspberry PI
which wraps the select/poll commands into a thread and calls a
\textsl{callback} function whenever an GPIO pin has been triggered.


\subsubsection{Timing with blocking I/O}
Blocking I/O (read, write, etc) can be used to time
the data coming in because the thread goes to sleep when it's waiting for
I/O but wakes up very quickly after new data has arrived.

In this example the blocking \texttt{read} command creates
the timing of the callback:
\begin{verbatim}
void run() {
       running = 1;
       while (running) {
              read(buffer); // blocking
              doCallback(buffer); // hand data to client
       }
}
\end{verbatim}


\subsubsection{Timing with Linux/pigpio timers}
Similar to threads one can create timers which are called at certain
intervals. These timers emit a Linux signal at a specified interval
and then this signal is caught by a global (static) function.
Generally it's \textsl{not recommended} to use timers for anything
which needs to be reliably sampled, for example ADC converters or
sensors with sampling rates higher than a few Hz. On the raspberry PI
use the pigpio library and its timer callbacks --- if needed at all.

\section{Conclusion}
Threads play a central role in real-time coding as,
together with blocking I/O, they establish
the callback interfaces. Every event handler
runs in a separate thread.

Callbacks are also used to signal the termination of
threads, which shows again the close relationship between threads
and callbacks.



\chapter{Realtime/event processing in Qt}

\section{Introduction}
\textbf{Qt} is a cross-platform windows development environment
for Linux, Windows and Mac.

Elements in Qt are \textsl{Widgets} which can contain
anything form plots, buttons or text fields. They are
classes. You can define your own widgets or use ready-made ones.

\section{Layout in Qt}

\begin{figure}[!hbt]
\begin{center}
\mbox{\includegraphics[width=0.5\textwidth]{qwtex}}
\end{center}
\caption{QT example layout
\label{qwtex}}
\end{figure}

There are different ways of declaring layout in Qt. One is
using a markup language which then has matching classes;
another is creating to use only C++ classes.
We show how to organise the layout using the second method.
This avoids having the learn an additional language and is
consistent with the general  trend to use code to declare the layout
in this and other frameworks (\textbf{SwiftUI}, for example).

This is an example how widgets are organised into
nested vertical and horizontal layouts (see Fig.~\ref{qwtex}
for the result).
\begin{verbatim}
// create 3 widgets
button = new QPushButton;
thermo = new QwtThermo; 
plot = new QwtPlot;

// vertical layout
vLayout = new QVBoxLayout;
vLayout->addWidget(button);
vLayout->addWidget(thermo);

// horizontal layout
hLayout = new QHBoxLayout;
hLayout->addLayout(vLayout);
hLayout->addWidget(plot);

// main layout
setLayout(hLayout);
\end{verbatim}


\section{Callbacks in Qt}
\subsection{Events from widgets}
In contrast to our low level callback mechanism using interfaces, Qt rather
directly calls methods in classes. The problem is that function pointers
cannot be directly used as a class has instance pointers to its local
data. So a method of a class needs to be combined with the instance
pointer. The Qt method ``connect'' does exactly that:
\begin{verbatim}
connect(button, &QPushButton::clicked,
        this, &Window::reset);
\end{verbatim}
The QPushButton instance \texttt{button} has a method called \texttt{clicked()} which is
called whenever the user clicks on the button. This is then forwarded to the
method \texttt{reset()} in the application Widget.


\subsection{Plotting realtime data arriving via a callback}
The general idea is to store the real-time samples from a callback in a
buffer and trigger a screen refresh at a lower rate. For example, we may
choose to replot the samples in the buffer every
40~ms because that fast enough for the user, whereas plotting the
whole buffer every sample would be too CPU-intensive.

A callback \texttt{addSample()} is called in real-time whenever
a sample has arrived:
%should probably change memmove to std::move or whatever
\begin{verbatim}
void Window::addSample( float v ) {
    // add the new input to the plot
    memmove( yData, yData+1, (plotDataSize-1) * sizeof(double) );
    yData[plotDataSize-1] = v;
}
\end{verbatim}
which stores the sample \texttt{v} in the shift buffer \texttt{yData}.

Then the screen refresh (which is slow) is done at
a lower and unreliable rate:
\begin{verbatim}
void Window::timerEvent( QTimerEvent * )
{
    curve->setSamples(xData, yData, plotDataSize);
    plot->replot();
    thermo->setValue( yData[0] );
    update();
}
\end{verbatim}

\texttt{update()} in the timer event-handler generates a
paint event and Qt then invokes the repaint
handler ``as soon as possible'' (which is to say, not in real-time) to repaint
the canvas of the widget:
\begin{verbatim}
void ScopeWindow::paintEvent(QPaintEvent *) {
        QPainter paint( this );

        paint.drawLine( ... )
}
\end{verbatim}

Note that neither the timer nor the \texttt{update()} function
is called in a reliable way but whenever Qt chooses to do it.
So Qt timers cannot be used to sample data but should
only be used for screen refresh and other non-time-critical
reasons.

\section{Conclusion}
Events in Qt are generated by user interaction, for example a button
press or moving the mouse. As before Qt provides a callback mechanism
via the \texttt{connect()} method. Callbacks from Qt timers may be used for
animations but must not be used for real-time events as Qt timers won't
guarantee a reliable timing.




\chapter{Realtime web server/client communication}

\section{Introduction}

There is a wide diversity of Web server / client applications
ranging from shopping baskets on vendor sites to social
media.

Generally it's easy to create dynamic content (see PHP or nodejs) and this is well
documented. However, feeding realtime data from C++ to a web page or
realtime button presses back to C++ is a bit more difficult.

It's important to recognise where \textsl{events} are generated: it is
always the client (web browser or mobile app) which triggers an event,
be it sending data over to the server or requesting data. It's
always initiated by the client.
%leave last sentence in for emphasis. It's counterintuitive and important

\section{REST}
The interface between a web client (browser or phone app) is usually
implemented as a Representational State Transfer Architectural (REST)
API by communicating via an URL on a web server. The requirements
for this API are very general and won't define the actual data format:
\begin{description}
\item[Uniform interface.] Any device connecting to the URL should
  get the same reply. No matter if a web page or mobile phone
  requests the temperature of a sensor the returned format must always be the same.
\item[Client-server decoupling.] The only information
  the client needs to know is the URL of the server to request data or send data.
\item[Statelessness.] Each request needs to include all the
  information necessary and must not depend on previous requests. For
  example a request to a buffer must not alter the
  buffer but just read from it so that another user reading the buffer
  shortly after receives the same data.
\end{description}
See \url{https://www.ibm.com/cloud/learn/rest-apis} for the complete
list of REST design principles.

\section{Data formats}
\subsection{Server $\to$ client: JSON (application/json)}
The most popular dataformat is JSON which is basically a
map of (nestable) key/value pairs:
\begin{verbatim}
{
    temperature: [20, 21, 20, 19, 17],
    steps: 100,
    comment: "all good!"
}
\end{verbatim}
Since JSON is human-readable text a web server can simply
generate that text send it over via http or https. There
is no difference except that the MIME format is \textquotesingle application/json\textquotesingle{}
instead of html.

\subsection{Client $\to$ server: POST (application/x-www-form-urlencoded)}
When a website or mobile app wants to send data back to the server it
needs to encode it in the form of a single text-line where the
key/value pairs are combined with \&-signs:
\begin{verbatim}
temperature=20&steps=100&comment=all+good%33
\end{verbatim}
The receiver then has the task to entangle this stream into a suitable data-format,
for example a map.


\section{Server}
On the Linux system a web server needs to be set up. There are
a variety of different options available but we are focusing here
on the ones which can be used for C++ communication (i.e. CGI).

\subsection{Web servers (http/https)}

\begin{itemize}
  \item NGINX: Easy to configure but very flexible web server. Pronounced ``Engine-X''.
  \item Apache: Hard to configure but safe option
  \item lighttpd: Smaller web server with a small memory footprint. Pronounced ``lighty''.
\end{itemize}
Note that it's possible to run different web servers at the
same time where they then act as proxies for a central web
server visible to the outside world. In particular nginx
makes it very easy to achieve this.

\begin{figure}[h]
\begin{center}
\includegraphics[width=\linewidth]{restdataflow}
\end{center}
\caption{FastCGI dataflow.\label{cgi}}
\end{figure}

\subsection{FastCGI}
FastCGI (see Fig~\ref{cgi}) is written in C++ and generates the entire
content of the http/https request. In particular here we generate
JSON packets server side which can then be processed by client JavaScripts.
For realtime applications JSON transmission is perfect because the client-side
JavaScript can request JSON packages and directly turn them into variables.

A fast CGI program is a UNIX commandline program which communicates with the web server
(nginx, Apache, \ldots) via a UNIX socket which in turn is a pseudo file located
in a temporary directory for example \texttt{/tmp/sensorsocket}.

The web server then maps certain http/https requests to
this socket. An example configuration for nginx looks like this:
\begin{verbatim}
       location /sensor/ {
          include        fastcgi_params;
          fastcgi_pass   unix:/tmp/sensorsocket;
        }
\end{verbatim}
If the user does a request via the URL \texttt{www.mywebpage.com/sensor/} then
nginx contacts the fastcgi program via this socket. The fastcgi program
then needs to return the content. Internally this will be a C++ callback
inside of the fastcgi program.

The C++ fastcgi API \url{https://github.com/berndporr/json_fastcgi_web_api} 
is wrapper around the quite cryptic fastcgi C library and we discuss
its callback handlers now.

\subsubsection{Server $\to$ client: JSON (application/json)}
The fastCGI callback expects a JSON string with the data
transmitted form the server to the client. There is helper
class JSONGenerator which generates the JSON data from various
C++ types:
\begin{verbatim}
class JSONcallback : public JSONCGIHandler::GETCallback {
public:
/**
* Gets the data and sends it to the webserver.
* The callback creates two JSON entries. One with the
* timestamp and one with the temperature from the sensor.
**/
virtual std::string getJSONString() {
      JSONCGIHandler::JSONGenerator jsonGenerator;
      jsonGenerator.add("epoch",(long)time(NULL));
      jsonGenerator.add("temperatures",temperatureArray);
      return jsonGenerator.getJSON();
    }
};
\end{verbatim} 

\subsection{Client $\to$ server: POST (application/x-www-form-urlencoded)}
Like in any GUI the client can press a button and create an event.  On
the client side this is packaged with jquery into an
``application/x-www-form-urlencoded'' stream and then sent over to the
server. In the server, \texttt{libcurl} is the standard tool for decoding
the application/x-www-form-urlencoded stream back into a C++ data
structure, for example a map.

Here, the server then receives the ``application/x-www-form-urlencoded'' data as a callback
called ``postArg'':
\begin{verbatim}
virtual void postString(std::string postArg) {
    auto m = JSONCGIHandler::postDecoder(postArg);
    float temp = atof(m["volt"].c_str());
    std::cerr << m["hello"] << "\n";
    sensorfastcgi->forceValue(temp);
}
\end{verbatim}
which is then decoded into key/value pairs in the form of a C++ \texttt{std::map}.

\section{Client code: javascript for websites}
Generally on the client side (= web page), HTML with embedded
\textsl{JavaScript} is used to generate realtime output/input without
reloading the page. JavaScript is \textsl{event driven} and has
callbacks so it's perfect for realtime applications. Use
\texttt{jQuery} to request and post JSON from/to the server.

For example here we request data from the server as a JSON
packet every second:

\begin{verbatim}
// callback when the JSON data has arrived
function getterCallback(result) {
  var temperatureArray = result.temperatures;
  // plot the array here
}

// timer callback (same idea as in Qt to define a refresh rate)
function getTemperature() {
  // get the JSON data
  $.getJSON("/data/:80",getterCallback);
}

// document ready callback
function documentReady() {
  // request new data from the server every second
  window.intervalId = setInterval(getTemperature , 1000);
}

// called when the web page has been loaded
$(document).ready( documentReady );
\end{verbatim}

Mobile phone programming in JAVA, Kotlin or Swift is also purely
callback driven as the JS code above and differs only in its syntax.

\section{Conclusion}
Events in web based communication are always triggered by the web
browser or the mobile app in exactly the same way as Qt does it, for
example by a button press. The same applies for animations where a
client-side timer requests data from the server. Thus, these client
side events either cause transmission of data from the web browser to
the web server or request data from the web server. Nowadays the protocol is
always http or https and a RESTful interface with JSON
being the most popular data format requested from the server.


\chapter{Setters}
In Fig.~\ref{gettersetters} we have seen that data flows from
the sensors to the C++ classes via \textsl{callbacks} then returns
from the inner C++ classes to motor or display outputs is via
\textsl{setters}. Setters are also used for setting configuration
parameters.

A setter is a simple method in a class, for example to set the
speed of a motor:
\begin{verbatim}
class Motor {
  /**
  * Set the Left Wheel Speed
  * @param speed between -1 and +1
  **/
  void setLeftWheelSpeed(float speed);
};
\end{verbatim}
Again as with callbacks it's important to \textsl{abstract} away from the
hardware, for example normalising the speed of the
motor between $-1$ and $+1$ and \textsl{hiding} away the complexity of the
PWM or GPIO ports in the class.

If a setter has more than one argument, in particular for
configuration, it's highly recommended to use a \textsl{struct} to set
the values. For example setting the parameters of the ADS1115:
\begin{verbatim}
/**
 * ADS1115 initial settings when starting the device.
 **/
struct ADS1115settings {

	/**
	 * I2C bus used (99% always set to one)
	 **/
	int i2c_bus = 1;

	/**
	 * I2C address of the ads1115
	 **/
	uint8_t address = DEFAULT_ADS1115_ADDRESS;
};

/**
 * Starts the data acquisition in the background and the
 * callback is called with new samples.
 * \param settings A struct with the settings.
 **/
void start(ADS1115settings settings = ADS1115settings() );
\end{verbatim}

If a setter sets large buffers then it's highly recommended
to allocate the memory in the constructor of the class and then call the setter by reference
while running. Use array types which convey their length, for example
std::array or a standard const array which implicitly carries their length.

\paragraph{Constant sampling rate output (audio, \ldots)}
There are many applications where the output device has a fixed
sampling rate, for example digital to analogue converters.  In this
case the C++ driver class will again have a blocking write-loop
periodically reading a buffer populated by the setter, which is
ideally always ahead of time.
% ^^^ That's still quite a complicated sentence, but I think it's right...
You need to decide what happens if no
fresh data has arrived, for example interpolating the output or
putting it on hold. Of course you can also implement a callback by the
audio write-loop to \textsl{request} samples but ultimately
the conflict between audio arriving and being dispatched needs to be
resolved.
% ^^^ Agian, check I didn't change your meaning too much,

\section{Conclusion}
Setters are simply methods which transmit an event back to the physical
device. Setters should, as callbacks, always hide the low level
complexity of the hardware device and receive normalised or physical
units.


\clearpage
\appendix
\chapter{License}
This work is licensed under the Creative Commons
Attribution-ShareAlike 4.0 International License. To view a copy of
this license, visit
\url{http://creativecommons.org/licenses/by-sa/4.0/} or
send a letter to Creative Commons, PO Box 1866, Mountain View, CA.


\end{document}
