\documentclass[xcolor=dvipsnames]{beamer}
\date{}
\title{GUI}
\author{Bernd Porr}
\begin{document}
\begin{frame}
\titlepage
\end{frame}




\begin{frame}[fragile]
  \frametitle{Web server/client communication: REST}
\begin{description}
\item[Uniform interface:] Any device connecting to the URL should
  get the same reply. No matter if a web page or mobile phone
  requests the temperature of a sensor the returned format must always be the same.
\item[Client-server decoupling:] The only information
  the client needs to know is the URL of the server to request data or send data.
\item[Statelessness:] Each request needs to include all the
  information necessary and must not depend on previous requests. For
  example a request to a buffer must not alter the
  buffer but just read from it so that another user reading the buffer
  shortly after receives the same data.
\end{description}
See \url{https://www.ibm.com/cloud/learn/rest-apis}
\end{frame}



\begin{frame}[fragile]
  \frametitle{Data format: JSON}

\begin{verbatim}
{
    temperature: [20, 21, 20, 19, 17],
    steps: 100,
    comment: "all good!"
}
\end{verbatim}
\end{frame}

\begin{frame}[fragile]
  \frametitle{Web servers}
\begin{itemize}
  \item NGINX: Easy to configure but very flexible web server. Pronounced ``Engine-X''.
  \item Apache: Hard to configure but safe option
  \item lighttpd: Smaller web server with a small memory footprint. Pronounced ``lighty''.
\end{itemize}
\end{frame}


\begin{frame}[fragile]
  \frametitle{Fast CGI}
A fast CGI program is a UNIX commandline program which communicates with the web server
(nginx, Apache, \ldots) via a UNIX socket which in turn is a pseudo file located
in a temporary directory for example \texttt{/tmp/sensorsocket}.

The web server then maps certain http/https requests to
this socket. An example configuration for nginx looks like this:
\begin{verbatim}
       location /sensor/ {
          include        fastcgi_params;
          fastcgi_pass   unix:/tmp/sensorsocket;
        }
\end{verbatim}
via the URL \texttt{www.mywebpage.com/sensor/}
\end{frame}


\begin{frame}[fragile]
  \frametitle{JSON C++ encoding with jsoncpp}
\begin{verbatim}
virtual std::string getJSONString() {
Json::Value root;
root["epoch"] = (long)time(NULL);
Json::Value values;
for(int i = 0; i < datasink->values.size(); i++) {
    values[i] = datasink->values[i];
}
root["values"]  = values;
Json::StreamWriterBuilder builder;
const std::string json_file = Json::writeString(builder, root);
return json_file;
};
\end{verbatim}
\end{frame}


\begin{frame}[fragile]
  \frametitle{JSON C++ decoding with jsoncpp}
  \begin{verbatim}
virtual void postString(std::string postArg) {
  const auto rawJsonLength = postArg.length();
  JSONCPP_STRING err;
  Json::Value root;
  Json::CharReaderBuilder builder;
  const std::unique_ptr<Json::CharReader>
    reader(builder.newCharReader());
  reader->parse(postArg.c_str(), postArg.c_str() + rawJsonLength, &root, &err)
  // do something with root
}
\end{verbatim}
  where \texttt{root} is a \texttt{std::map}.
\end{frame}


\begin{frame}[fragile]
  \frametitle{Web browser: javascript}

  \begin{verbatim}
function getterCallback(result) {
  var temperatureArray = result.temperatures;
}

function getTemperature() {
  $.getJSON("/sensor/:80",getterCallback);
}

function documentReady() {
  window.intervalId = setInterval(getTemperature , 1000);
}

$(document).ready( documentReady );
\end{verbatim}
\end{frame}

\end{document}



\end{document}
