\documentclass[xcolor=dvipsnames]{beamer}
\title{Realtime Embedded Coding: Threads}
\date{}
\author{Bernd Porr}
\begin{document}
\begin{frame}
\titlepage
\end{frame}

\begin{frame}[fragile]
\frametitle{Introduction}
\begin{enumerate}
\item Load balancing (using all CPU cores)
\item Endless loops with blocking I/O or GPIO wakeups
  to establish \textbf{precise timing} for callbacks.
\item \textbf{Asynchronous execution} of time-consuming tasks
  with a callback after the task has completed.
\end{enumerate}
\end{frame}


\begin{frame}[fragile]
\frametitle{Thread and worker}
A thread is just a \textsl{wrapper} for the actual method
which is running independently.

The method being run in the thread
is often called a \textsl{worker}.
\end{frame}


\begin{frame}[fragile]
\frametitle{Creating threads}
In C++ a worker is a method within
a class:
\begin{verbatim}
uthread = new std::thread(&MyClassWithAThread::run, this);
\end{verbatim}
where \texttt{MyClassWithAThread} is a class containing the method ``run'':
\begin{verbatim}
class MyClassWithAThread {
        void run() {
                // ... hard work is done here
                doCallback(result); // hand the result over
        }
}
\end{verbatim}
The ``\&'' signifies a functor. You can also use a lambda function:
\begin{verbatim}
uthread = std::thread([this](){run();});
\end{verbatim}
\end{frame}


\begin{frame}[fragile]
\frametitle{Lifetime of a thread}
Threads terminate simply once the worker has finished its job.
To wait for the termination
of the thread use the ``join()'' method:
\begin{verbatim}
        void stop() {
                uthread->join();
                delete uthread;
        }
\end{verbatim}
\end{frame}



\begin{frame}[fragile]
\frametitle{Timed callbacks with blocking I/O and endless loops}
Threads with endless loops are often used in conjunction with blocking
I/O which provide the timing:
\begin{verbatim}
void run() {
       running = true;
       while (running) {
              read(buffer); // blocking
              doCallback(buffer); // hand data to client
       }
}
\end{verbatim}
\end{frame}



\begin{frame}[fragile]
  \frametitle{Timing without blocking I/O: \texttt{select()}}
Non-blocking I/O can be turned into blocking I/O with \texttt{select()}
which waits till data has become available.
\begin{verbatim}
FD_ZERO(&rdset);
FD_SET(fileno,&rdset);
struct timeval timeout;
timeout.tv_sec = 0;
timeout.tv_usec = 500000;
int ret = select(fileno+1,&rdset,NULL,NULL,&timeout);
if (ret < 0) return ret;
// non-blocking I/O read
ret = read(fileno,buffer,bytes_per_sample * n_chan);
\end{verbatim}
\end{frame}


\begin{frame}[fragile]
  \frametitle{Timing without blocking I/O: \texttt{poll()}}
  \texttt{poll()} is similar to \texttt{select()} but
  has less temporal resolution (ms) to wake up a thread.
  Here used to wake up a thread after a GPIO pin has been
  triggered in \texttt{/sys}:
\begin{verbatim}
int SysGPIO::gpio_poll(int gpio_fd, int timeout) const
{
  struct pollfd fdset[1] = {};
  int nfds = 1;
  char buf[MAX_BUF] = { 0 };
  fdset[0].fd = gpio_fd;
  fdset[0].events = POLLPRI;

  int rc = poll(fdset, nfds, timeout);

  if (fdset[0].revents & POLLPRI) {
    (void)read(fdset[0].fd, buf, MAX_BUF);
  }
  return rc;
}
\end{verbatim}
\end{frame}


\begin{frame}[fragile]
\frametitle{Running/stopping workers with endless loops}
The flag \texttt{running} which is controlled by the main program and is set to zero to terminate
the thread:
\begin{verbatim}
        void stop() {
                running = false; // <----- HERE!!
                uthread->join();
                delete uthread;
        }
\end{verbatim}
Note that \texttt{join()} is a \textbf{blocking} operation and needs
to be used with care not to lock up the main program. You probably
only need it when your program is terminating.  See
\url{https://github.com/berndporr/rpi_AD7705_daq} for an example.
\end{frame}


\begin{frame}[fragile]
\frametitle{Timing within threads: Timing with blocking I/O}

In this example the blocking \texttt{read} command creates
the timing of the callback:
\begin{verbatim}
void run() {
       running = 1;
       while (running) {
              read(buffer); // <--- waits for data
              doCallback(buffer); // hand data to client
       }
}
\end{verbatim}
\end{frame}

\begin{frame}[fragile]
\frametitle{Timing with Linux/pigpio timers}
As a \textsl{last resort} one can use a timer. Similar to threads one can
create timers which are then called at certain intervals. As with threads
timers should be \textsl{hidden} within a C++ class as
\textsl{private} members which then trigger \textsl{public callbacks}
via C++ callback mechanisms as described above.

Generally it's \textsl{not recommended}
to use timers for anything which needs to be reliably sampled, for
example ADC converters or sensors with sampling rates higher than a
few Hz. On the raspberry PI use the pigpio library and its timer
callbacks --- if needed at all.
\end{frame}

\begin{frame}[fragile]
  \frametitle{Summary}
  \begin{itemize}
  \item Threads do load balancing.
  \item Threads create timing by using blocking I/O.
  \item Threads prevent programs from locking up.
  \end{itemize}
  \ldots and remember: threads running in classes and communicate
  via callbacks!
\end{frame}

\end{document}
