\documentclass[xcolor=dvipsnames]{beamer}
\date{}
\title{GUI}
\author{Bernd Porr}
\begin{document}
\begin{frame}
\titlepage
\end{frame}

\begin{frame}[fragile]
  \frametitle{Introduction}

\begin{itemize}
\item \textbf{Qt} is a cross-platform windows development environment for
Linux, Windows and Mac.

\item QT works with callbacks as introduced above
but they are wrapped in a QT-specific signal/slot concept.

\item Elements in Qt are \textsl{Widgets} which can contain anything form
  plots, buttons or text fields. They are classes.
  
\end{itemize}
  
\end{frame}



\begin{frame}[fragile]
  \frametitle{Layout}

\begin{figure}[!hbt]
\begin{center}
\mbox{\includegraphics[width=0.5\textwidth]{qwtex}}
\end{center}
\caption{QT example layout
\label{qwtex}}
\end{figure}

Widgets are organised into
nested vertical and horizontal layouts
for the result).
\end{frame}



\begin{frame}[fragile]
  \frametitle{Layout II}

\begin{verbatim}
// create 3 widgets
button = new QPushButton;
thermo = new QwtThermo; 
plot = new QwtPlot;

// vertical layout
vLayout = new QVBoxLayout;
vLayout->addWidget(button);
vLayout->addWidget(thermo);

// horizontal layout
hLayout = new QHBoxLayout;
hLayout->addLayout(vLayout);
hLayout->addWidget(plot);

// main layout
setLayout(hLayout);
\end{verbatim}
\end{frame}



\begin{frame}[fragile]
\frametitle{Events from widgets}
In contrast to our low level callback mechanism using interfaces, Qt rather
directly calls methods in classes. The problem is that function pointers
cannot be directly used as a class has instance pointers to its local
data. So a method of a class needs to be combined with the instance
pointer. The Qt method ``connect'' does exactly that:
\begin{verbatim}
connect(button, &QPushButton::clicked,
        this, &Window::reset);
\end{verbatim}
The QPushButton instance \texttt{button} has a method called \texttt{clicked()} which is
called whenever the user clicks on the button. This is then forwarded to the
method \texttt{reset()} in the application Widget.
\end{frame}


\begin{frame}[fragile]
\frametitle{Plotting realtime data arriving via a callback}
The general idea is to store the real-time samples from a callback in a
buffer and trigger a screen refresh at a lower rate. For example, we may
choose to replot the samples in the buffer every
40~ms because that fast enough for the user, whereas plotting the
whole buffer every sample would be too CPU-intensive.

A callback \texttt{addSample()} is called in real-time whenever
a sample has arrived:
%should probably change memmove to std::move or whatever
\begin{verbatim}
void Window::addSample( float v ) {
    // add the new input to the plot
    std::move( yData, yData + plotDataSize - 1, yData+1 );
    yData[0] = v;
}
\end{verbatim}
which stores the sample \texttt{v} in the shift buffer \texttt{yData}.
\end{frame}


\begin{frame}[fragile]
\frametitle{Plotting realtime data arriving via a callback}
Then the screen refresh (which is slow) is done at
a lower and unreliable rate:
\begin{verbatim}
void Window::timerEvent( QTimerEvent * )
{
    curve->setSamples(xData, yData, plotDataSize);
    plot->replot();
    update();
}
\end{verbatim}
\end{frame}


\begin{frame}[fragile]
\frametitle{Plotting realtime data arriving via a callback}
\texttt{update()} in the timer event-handler generates a
paint event and Qt then invokes the repaint
handler ``as soon as possible'' (which is to say, not in real-time) to repaint
the canvas of the widget:
\begin{verbatim}
void ScopeWindow::paintEvent(QPaintEvent *) {
        QPainter paint( this );

        paint.drawLine( ... )
}
\end{verbatim}
\end{frame}


\begin{frame}[fragile]
\frametitle{Plotting realtime data arriving via a callback}
Note that neither the timer nor the \texttt{update()} function
is called in a reliable way but whenever Qt chooses to do it.
So Qt timers cannot be used to sample data but should
only be used for screen refresh and other non-time-critical
reasons.
\end{frame}







\end{document}
